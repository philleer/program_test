% Latex code segment for flow chart and shell symbols


\documentclass[a4paper,UTF8]{ctexart}
\usepackage{tikz}
\usetikzlibrary{shapes,arrows,chains}
\usepackage{lscape}
\usepackage{booktabs}
\usepackage{geometry}
\usepackage{tabularx}
\usepackage{amssymb}
\usepackage{amsmath}

\begin{document}
\pagestyle{empty}

shell命令执行过程完全解析,无论是通配符~\ref{wildcard}~、元字符~\ref{metachar}~或者转义字符也好,都有其特定含义和用法,工欲善其事必先利其器,需要我们详细解剖整个过程。

\begin{figure}[htbp]
\centering

\tikzstyle{startstop} = [rectangle,rounded corners,minimum width=3cm,minimum height=1cm,text centered,draw=black,fill=white!30]
\tikzstyle{io} = [trapezium,trapezium left angle=70,trapezium right angle=110,minimum width=3cm,minimum height=1cm,text centered,draw=black,fill=white!30]
\tikzstyle{process} = [rectangle,minimum width=3cm,minimum height=1cm,text centered,text width=3cm,draw=black,fill=white!30]
\tikzstyle{decision} = [diamond,minimum width=3cm,minimum height=1cm,text centered,draw=black,fill=white!30]
\tikzstyle{arrow} = [thick,->,>=stealth]

\begin{tikzpicture}[node distance=2cm]

% \node (start) [startstop] {Start};
% \node (input1) [io,below of=start] {Input};
% \node (process1) [process,below of=input1] {Process 1};
% \node (decision1) [decision,below of=process1,yshift=-0.5cm] {Decession 1};
% \node (process2a) [process,below of=decision1,yshift=-0.5cm] {Process 2aaa};
% \node (process2b) [process,right of=decision1,xshift=2cm] {Process 2b};
% \node (out1) [io,below of=process2a] {Output};
% \node (stop) [startstop,below of=out1] {Stop};
% 
% \draw [arrow] (start) -- (input1);
% \draw [arrow] (input1) -- (process1);
% \draw [arrow] (process1) -- (decision1);
% \draw [arrow] (decision1) -- node[anchor=east] {yes} (process2a);
% \draw [arrow] (decision1) -- node[anchor=south] {no} (process2b);
% \draw [arrow] (process2b) |- (process1);
% \draw [arrow] (process2a) -- (out1);
% \draw [arrow] (out1) -- (stop);

\node (start) [process] {读取命令行};
\node (process1) [process,right of=start,xshift=2cm] {历史命令替换};
\node (process2) [process,right of=process1,xshift=2cm] {别名替换};
\node (process3) [process,right of=process2,xshift=2cm] {花括号扩展};
\node (process4) [process,below of=process3] {波浪号扩展};
\node (process5) [process,left of=process4,xshift=-2cm] {I/O重定向};
\node (process6) [process,left of=process5,xshift=-2cm] {变量替换};
\node (process7) [process,left of=process6,xshift=-2cm] {命令替换};
\node (process8) [process,below of=process7] {单词解析};
\node (process9) [process,right of=process8,xshift=2cm] {文件名生成};
\node (process10) [process,right of=process9,xshift=2cm] {引用字符处理};
\node (process11) [process,right of=process10,xshift=2cm] {进程替换};
\node (process12) [process,below of=process11] {环境处理};
\node (process13) [process,left of=process12,xshift=-2cm] {执行命令};
\node (process14) [process,left of=process13,xshift=-2cm] {跟踪执行过程};
\node (stop) [process,left of=process14,,xshift=-2cm] {输出结果};

\draw [arrow] (start) -- (process1);
\draw [arrow] (process1) -- (process2);
\draw [arrow] (process2) -- (process3);
\draw [arrow] (process3) -- (process4);
\draw [arrow] (process4) -- (process5);
\draw [arrow] (process5) -- (process6);
\draw [arrow] (process6) -- (process7);
\draw [arrow] (process7) -- (process8);
\draw [arrow] (process8) -- (process9);
\draw [arrow] (process9) -- (process10);
\draw [arrow] (process10) -- (process11);
\draw [arrow] (process11) -- (process12);
\draw [arrow] (process12) -- (process13);
\draw [arrow] (process13) -- (process14);
\draw [arrow] (process14) -- (stop);

\end{tikzpicture}
%\caption{shell命令解析流程}
\end{figure}

\newgeometry{left=2.5cm,right=2.5cm,,top=2.5cm,bottom=2.5cm}

\begin{landscape}
\begin{table}[p]
	\renewcommand{\arraystretch}{1.5}
	\newcommand{\tc}[1]{\multicolumn{1}{c}{#1}}
	\newcommand{\tl}[1]{\multicolumn{1}{l}{#1}}
	\newcommand{\tr}[1]{\multicolumn{1}{r}{#1}}
	\centering
	\caption[wildcard]{Shell通配符}
	\label{wildcard}
	\vspace{0.8em}\centering
	\begin{tabular}{cll}
		\toprule[2pt]
		\textbf{字符} & \tc{\textbf{描述}} & \tc{\textbf{示例}}\\[2pt]\midrule[0.8pt]
		$\ast$ & 匹配0或多个字符 & a$\ast$b表示a与b之间可以有任意长度的任意字符,包括0,如ab,a01b,axyzb等\\[8pt]
		? & 匹配任意单个字符 & a?b表示a与b之间必须也只能有一个字符,但可以是任意字符,如aab,a0b等\\[8pt]
		[list] & 匹配list中任意单个字符 & a[xyz]b表示a与b之间只能为x或y或z,匹配axb,ayb,azb等\\[8pt]
		[!list] & 匹配list之外的任意单个字符 & a[!0-9]b表示a与b之间只能有一个非阿拉伯数字,匹配axb,a-b,aZb等\\[8pt]
		[c1-c2] & 匹配c1-c2之间的任意单个字符,要写成递增 & a[0-9]b表示a与b之间只能有一个阿拉伯数字,如a0b,a8b等\\[8pt]
		\{string1,string2,...\} & 匹配穷举列表中的其中一个字符串 & a\{abc,xyz,123\}b表示a与b之间只能是abc或xyz或123三个字符串之一\\[8pt]
		\bottomrule[2pt]
	\end{tabular}
\end{table}
\end{landscape}

\begin{table}[htbp]
	\renewcommand{\arraystretch}{1.5}
	\newcommand{\tc}[1]{\multicolumn{1}{c}{#1}}
	\centering
	\caption[metachar]{Shell元字符列表}
	\label{metachar}
	\vspace{0.8em}\centering
	\begin{tabular}{cl}
		\toprule[2pt]
		\textbf{字符} & \tc{\textbf{描述}}\\[2pt]
		\midrule[0.8pt]
		IFS & 由<space>或<tab>或<enter>三者之一组成\\[8pt]
		CR & 由<enter>产生\\[8pt]
 		= & 设定变量\\[8pt]
 		\$ & 作变量或运算替换\\[8pt]
 		> & 重定向输出stdout\\[8pt]
 		< & 重定向输入stdin\\[8pt]
		$\mid$ & 管道命令\\[8pt]
 		\& & 重定向file descriptor,或将命令置于后台运行\\[8pt]
 		(\ ) & 将其内的命令置于nested subshell执行,或用于运算或命令替换\\[8pt]
 		\{\ \} & 将其内的命令置于non-named function中执行,或用在变量替换的界定范围\\[8pt]
 		; & 在前一个命令结束时,忽略其返回值,继续执行下一个命令\\[8pt]
 		\&\& & 在前一个命令结束时,若返回值为true,则继续执行下一个命令\\[8pt]
 		$\mid\mid$ & 在前一个命令结束时,若返回值为false,则继续执行下一个命令\\[8pt]
 		! & 执行history列表中的命令\\[8pt]
 		\bottomrule[2pt]
	\end{tabular}
\end{table}

\cleardoublepage

\begin{table}[htbp]
\renewcommand{\arraystretch}{1.5}
\newcommand{\tc}[1]{\multicolumn{1}{c}{#1}}
\centering
\caption[escape]{Shell转义符}
\label{escape}
\vspace{1.0em}
\centering
	\begin{tabular}{cl}
		\toprule[2pt]
			\textbf{字符} & \tc{\textbf{描述}}\\[2pt]
		\midrule[0.8pt]
			$\`$ $\'$ & 又称硬转义,其内部所有shell元字符、通配符都失效,需要注意中间不允许出现单引号$\'$\\[8pt]
			$\`$ $\`$ $\'$ $\'$ & 又称软转义,其内部只允许出现特定的shell元字符:\$用于参数替换,$\`$用于命令代替\\[8pt]
			$\backslash$ & 又称转义,去除其后紧跟的元字符或通配符的特殊含义\\[8pt]
		\bottomrule[2pt]
	\end{tabular}
\end{table}

\begin{table}[htbp]
	\renewcommand{\arraystretch}{1.5}
	\newcommand{\tc}[1]{\multicolumn{1}{c}{#1}}
	\centering
	\caption[examples]{Shell一些转义字符及其含义}
	\label{examples}
	\vspace{1.0em}
	\centering
	\begin{tabular}{rl}
		\toprule[2pt]
		\tc{\textbf{字符}} & \tc{\textbf{描述}}\\[2pt]
		\midrule[0.8pt]
		$\backslash$a & 输出警告音\\[8pt]
		$\backslash$b & 退格,向左删除一个字符\\[8pt]
		$\backslash$e \ $\backslash$E & <ESC>\\[8pt]
		$\backslash$f & 换页符\\[8pt]
		$\backslash$n & 换行\\[8pt]
		$\backslash$r & 回车\\[8pt]
		$\backslash$t & 制表符,即TAB\\[8pt]
		$\backslash$v & 垂直制表符\\[8pt]
		$\backslash\backslash$ & 输入$\backslash$本身\\[8pt]
		$\backslash$ $\'$ & 输入单引号$\'$本身\\[8pt]
		$\backslash$ $\'$ $\'$ & 输入单引号$\'$ $\'$本身\\[8pt]
		$\backslash$nnn & 按照八进制ASCII码表输出相应字符,n表示一位八进制数\\[8pt]
		$\backslash$xHH & 按照十六进制ASCII码表输出相应字符,h表示一位十六进制数\\[8pt]
		$\backslash$cx & 取消输出行末的换行符\\[8pt]
		\bottomrule[2pt]
	\end{tabular}
\end{table}

\clearpage

\end{document}


% @article 期刊文章
% @book 有明确出版商的书
% @ booklet 没有明确出版商或赞助商的印刷品
% @conference 回忆文章,与inproceeding相同
% @inbook 书的一部分,可以是章节等
% @incollection 书的一部分,有小标题
% @inproceeding 回忆文章
% @manual 技术文档
% @masterthesis 硕士论文
% @misc 大杂烩,没有其他合适的类型时使用
% @phdthesis 博士论文
% @proceeding 会议记录
% @techreport 由学校或其他机构出版的报告,通常含有编号
% @unpublished 有作者和标题单位正式出版的文件
